\documentclass[12pt]{scrartcl}

\usepackage[utf8]{inputenc}
\usepackage[ngerman]{babel}
\usepackage[T1]{fontenc}

\usepackage[backend=bibtex,citestyle=verbose-ibid, sorting=none]{biblatex}
\addbibresource{mybib.bib}

\usepackage[hidelinks]{hyperref}
\usepackage[a4paper, lmargin=0.13\paperwidth, rmargin=0.13\paperwidth, tmargin=0.08\paperheight, bmargin=0.1111\paperheight]{geometry} %margins
\linespread{1.5}

\begin{document}

\begin{titlepage}
    \title{Innenpolitik unter Willy Brandt: Radikalenerlass}
    \date{09. August 2022}
    \author{Patrick Zauner}
    \maketitle
    \begin{center}
    T-GEISTSOZ-109228 – Modulteilprüfung schriftlich - Grundlagen der Geschichtswissenschaft
    \end{center}
    \thispagestyle{empty}
\end{titlepage}

\tableofcontents
\setcounter{page}{1}

\newpage


\section{Hintergrund des Radikalenerlasses}

Das Motto der SPD unter der Idee einer neuen Ostpolitik, welche als Ergänzung zur Westpolitik unter der Ära Adenauers gesehen werden muss, \footcite[][S. 357]{gebhardt_handbuch_2001} war eine »Politik der kleinen Schritte«. \footcite[][S. 358]{gebhardt_handbuch_2001}

Zwar hatte die neue Ostpolitik nicht nur die Besserstellung Deutschlands im Hinblick auf Verträge mit der Sowjetunion zum Ziel, sondern auch eine Aussöhnung und Bitte um Vergebung gegenüber Polen aufgrund der Invasion Hitlers. \footcite[][S. 378f]{gebhardt_handbuch_2001} Dies stand seit dem Kniefall Brandts vor dem Ehrendenkmal der Opfer des jüdischen Ghettos in Warschau im Winter 1970 zweifelsfrei fest. Mit Blick auf das gespaltene Deutschland und die DDR stellte man bereits in der Regierungserklärung der sozialliberalen Koalition aus SPD und FDP fest, dass eine völkerrechtliche Anerkennung der DDR aus Sicht der Bundesregierung nicht in Betracht komme. \footcite[][S. 33]{rodder_bundesrepublik_2004}


Der Radikalenerlass steht insgesamt jedoch unter der langfristigen Idee der neuen Ostpolitik, die auch eine Nähe zur Sowjetunion und Einigung erfordert. Auf den Druck der Union, die eine zu große Nähe zu kommunistischen Gedanken durch die Nähe zum Osten nahelegt, kann so auch der Radikalenerlass als ein Zugeständnis der SPD zum demokratischen Deutschland verstanden werden. So dürfe nicht die Wiedervereinigung aus den Augen verloren werden und schon das Verhandeln mit der SED überhaupt unterlag dem moralischen Zweifel, ob eine gleichberechtigte Verhandlung zwischen einer demokratisch legitimierten Regierung und einem »kommunistischen Zwangsregime« überhaupt möglich sei. \footcite[][S. 387]{gebhardt_handbuch_2001}
Ebenso mahnten Kritiker aus der Union, dass »wer dem linken Totalitarismus eine Gleichberechtigung zugestehe«, Gefahr laufe, »dass Linksextreme die Bundesrepublik unterwanderten«. \footcite[][S. 387]{gebhardt_handbuch_2001}

\section{Umsetzung und Wahrnehmung}

Mit dem Beschluss zur »Beschäftigung von rechts- und linksradikalen Personen im öffentlichen Dienst« \footcite[][S. 342]{noauthor_ministerialblatt_1972} für welchen sich auch die Benennungen »Radikalenerlass« und »Extremistenbeschluss« fanden und von »Berufsverbote[n]« \footcite[][S. 409]{gebhardt_handbuch_2001} die Rede war, zeigte sich auch eine Dichotomie: Einerseits soll die Verfassungstreue gewahrt und eine Unterwanderung von links, wie von der Union befürchtet, ausgeschlossen werden, andererseits wurde Kritik am »antidemokratischen Charakter« des Beschlusses laut \footcite[][S. 409]{gebhardt_handbuch_2001} - erinnerte er nicht zuletzt auch an eine Fortsetzung der politischen Strafverfolgung des Adenauererlasses.

Die Idee des Radikalenerlasses bestand darin, auszuschließen, dass verfassungsfeindliche Personen für die Bundesrepublik arbeiteten. So galt der Beschluss für sämtliche Beamte als auch in Teilen für andere Angestellte im öffentlichen Dienst als auch in einer leicht abgeänderten Form für alle Bewerber, deren Verfassungskonformität per Regelabfrage bei Stellen des Verfassungsschutzes geprüft werden sollte.

Dabei gab es eine Unterscheidung zwischen bereits Verbeamteten, für welche relevant ist, ob diesen ein verfassungsfeindliches Handeln nachgewiesen werden kann, während für sich neu Bewerbende lediglich zählt, ob diese »verfassungsfeindliche Aktivität[en]« entwickeln. \footcite[Vgl. Abs. 2.1.1 und 2.2][S. 342]{noauthor_ministerialblatt_1972} Dabei legte die ursprüngliche Formulierung einen stärkeren Fokus auf die Vergangenheit, weil hier lediglich gefordert war, dass ein Bewerber eine verfassungsfeindliche Aktivität entwickelt hat. Diese Änderung dürfte Brandts Initiative geschuldet sein (so ist etwa ein Ergebnisprotokoll die einzige Dokumentation des Treffens) und war die einzige am Text der über die Ministerpräsidentenkonfernz beschlossenen Regelung. \footcite[Vgl.][S. 340f]{rigoll_staatsschutz_2013}

Auch war der genaue Wortlaut nie bekannt, wurde er schlussendlich erst von den einzelnen Ländern übernommen und in den Ministerialblättern publiziert. \footcite[][S. 342]{rigoll_staatsschutz_2013}

Dem Beschluss voraus ging die politische Diskussion, inwieweit die Parteimitgliedschaft allein die These begründen kann, eine Person hege das Interesse, die demokratische Grundordnung zu beseitigen und damit nicht als Beamter, der für die Verfassung einzustehen hat, tätig werden zu können. Wehner, Vorsitzender der SPD-Bundestagsfraktion, begegnete diesem Postulat der Union mit einem Artikel in der Augsburger Allgemeinen, in welchem er argumentierte, dass, wer sich nicht dem Vorwurf der Gesinnungsschnüffelei stellen will, auch nicht aus der bloßen Mitgliedschaft, die - so meinte Justizminister Jahn - nur intensiver Ausdruck der eigenen Gesinnung sei, dies auf alle Bediensteten wie Bewerber ausweiten oder eine Abschwächung des Diskriminierungsverbots erwirken müsse. Die Reaktion der Union hierauf war die Bereitschaft, eine Grundgesetzänderung nicht auszuschließen, wenn somit DKP-Mitglieder vom öffentlichen Dienst ausgeschlossen wörden. \footcite[][S. 337]{rigoll_staatsschutz_2013}

Effektive Beachtung fand die Argumentation Wehners schlussendlich nicht, wenn man die Unterscheidung zwischen bereits Bediensteten und Bewerbern im Beschluss betrachtet. In einem späteren Fall der Benachteiligung einer Lehrerin durch den Radikalenbeschluss bestätigte der europäische Gerichtshof für Menschenrechte den Gedankengang, dass mindestens aufgrund der Funktion der angestrebten Position eine Differenzierung der unterschiedlichen Bereiche des öffentlichen Dienstes vorzunehmen sei. \footcite[][S. 275]{edgar_wolfrum_verfassungsfeinde_2022}

Hintergrund für die Wahrnehmung des »antidemokratischen Charakter[s]« des Beschlusses war nicht der erste Teil, welcher die Verfassungstreue Verbeamteter forderte, sondern vielmehr die Regelabfrage der Bewerberinnen und Bewerber, welche nicht nur intransparent war, sondern auch ein Grundmisstrauen gegenüber der eigenen Bürgerinnen und Bürger bedeutete. \footcite[][S. 410]{gebhardt_handbuch_2001} Auch ist aus rechtsstaatlichen Gesichtspunkten wie dem Rückwirkungsverbot von Gesetzen, die Auslegung auf bereits verbeamtete Personen, welchen etwa aufgrund vergangener Parteimitgliedschaften das Postulat der Verfassungsfeindlichkeit zugewiesen wurde, kritisch zu bewerten. Ebenso lässt sich diese Idee auch auf begangene Sachverhalte von Bewerberinnen und Bewerber ausweiten, welche vor dem Beschluss stattfanden, weil ihnen nicht die notwendigen Informationen über sämtliche Konsequenzen bekannt gewesen sein konnten.

Kritisch ist ebenso die von Grund auf offene Auslegungsmöglichkeit. Einerseits lässt sich der Beschluss liberal auffassen, bietet jedoch auch die Interpretationsmöglichkeit einer strikten und undurchsichtigen Auslegung. \footcite[][S. 339]{rigoll_staatsschutz_2013} So war weder eindeutig, welche Aktivitäten oder Organisationen als verfassungsfeindlich zählten, noch war im Fall der Zuschreibung der Verfassungsfeindlichkeit, ungeachtet der Gründe, von einer Entscheidung im Sinne des Bewerbers auszugehen, »rechtfertigten doch die mitgliedschaftsinduzierten Zweifel an der Verfassungstreue „in der Regel“\footcite[Vgl.][S. 342]{noauthor_ministerialblatt_1972} die Ablehnung.« \footcite[][S. 339]{rigoll_staatsschutz_2013}

\subsection{Transparenz bei der Ablehnung von Bewerbern}

Um das Finden der Radikalen zu vereinfachen und gewissermaßen zu vereinheitlichen, gab die Innenministerkonferenz im April 1972 eine Liste mit den Parteien »DKP, SDAJ, KPD/ML sowie drei neonazistische Gruppierungen (...), darunter die NPD [befanden]«. Öffentlich bestritten wurde eine Existenz solcher Listen. \footcite[][S. 346]{rigoll_staatsschutz_2013} 

Unklar allerdings war man sich in der Frage, ob abgelehnten Bewerbern die Gründe mitgeteilt werden sollten, sodass eine Einzelfallprüfung, welche vorab gefordert war, überhaupt angestrebt werden konnte. Das allerdings wurde von der Mehrheit der Innenminister abgelehnt, da eine Mitteilung der Gründe eine Bevorteilung gegenüber anderen Gründen wäre. \footcite[][S. 346]{rigoll_staatsschutz_2013} Andererseits ließen sich hier Gründe differenzieren, etwa zwischen »formalen« und »qualitativen« Gründen. Dies wäre wichtig, um die Möglichkeit, den Rechtsweg zu bestreiten, überhaupt zu gewährleisten. Ebenfalls ließe sich die Ansicht des Bremer Senats vertreten, wonach die Ablehnung alleine aufgrund der formalen Gründe, wozu allerdings auch eine Parteimitgliedschaft gehöre, dann kommuniziert werde. Diese Regelung verfolgt die Idee, dem Bewerber die Gelegenheit zu geben, die Zweifel gegebenenfalls zu widerlegen. \footcite[][S. 346]{rigoll_staatsschutz_2013} Da jedoch auch qualitative Gründe notwendig für die Einstellung sind, werde man, sofern diese nicht erfüllt sind, keine weiteren Details nennen.

\subsection{Das Urteil des Bundesverfassungsgerichts zum Radikalenerlass}

Die Argumentation hinter der dennoch bestehenden Rechtmäßigkeit nimmt jedoch nicht die vergangenen Umstände als unmittelbaren Grund für die Ablehnung, sondern leitet mittelbar ab, dass die gesamte Gesinnung der Person unvereinbar mit der Treuepflicht des Beamtenstatus sei. Denn so erfordere die Treuepflicht ebenfalls vom Beamten, »dass er sich eindeutig von Gruppen und Bestrebungen distanziert, die diesen Staat, seine verfassungsmäßigen Organe und die geltende Verfassungsordnung angreifen, bekämpfen und diffamieren« \footcite[][Abs. 2]{noauthor_bverfge_nodate}

Dabei gelte dieser Grundsatz eben auch für sämtliche Beamtenverhältnisse, wie auch für solche auf Zeit, Probe oder im Falle von Referendaren auch auf Widerruf. Ebenso gelte es gar, wenngleich in abgeschwächter Form, auch für weitere Angestellte im öffentlichen Dienst. \footcite[][S. 268]{edgar_wolfrum_verfassungsfeinde_2022}
So stellte das Bundesverfassungsgericht fest, dass »wenn auch an die Angestellten im öffentlichen Dienst weniger hohe Anforderungen als an die Beamten zu stellen sind, (...) sie gleichwohl dem Dienstherrn Loyalität und die gewissenhafte Erfüllung ihrer dienstlichen Obliegenheiten [schulden]«. Da auch sie im Dienstverhältnis zum Staat stehen dürfen auch sie die Verfassungsordnung nicht angreifen und können ebenso bei grober Verletzung der Dienstpflichten fristlos entlassen werden, sowie ihre Einstellung abgelehnt werden, »wenn damit zu rechnen ist, daß sie ihre mit der Einstellung verbundenen Pflichten nicht werden erfüllen können oder wollen.« \footcite[][Abs. 7]{noauthor_bverfge_nodate}

Die Idee des Berufsverbots und damit einen Widerspruch zu Artikel 12 des Grundgesetzes, das allen Deutschen das Recht der Berufsfreiheit gewährt, kann das Bundesverfassungsgericht nicht anerkennen. So sprechen »zwingende Gründe des Allgemeinwohls« für die Kopplung an die Gewähr der der Verfassungstreue durch den Bewerber. Insoweit mindestens die Funktionsfähigkeit des Staates, der auch einen Selbsterhaltungszweck verfolgt, »vom Beamtenkörper abhängt«, sei die Zulassungsvoraussetzung der Gewähr der Verfassungstreue erforderlich. \footcite[][Ziffer III, Abs. 4b]{noauthor_bverfge_nodate}

Daher sei auch das »politische Schlag- und Reizwort« des Berufsverbots »völlig fehl am Platz«, das »offensichtlich nur politische Emotionen wecken soll«. \footcite[][Ziffer III, Abs. 4b]{noauthor_bverfge_nodate}

Relativierend konkretisierte man jedoch, da juristische Referate auch für einen freien Beruf ein Beamtenverhältnis auf Widerruf voraussetzen, die Notwendigkeit des Beamtenverhältnisses. Sofern ein Vorbereitungsdienst ebenso für einen freien Beruf erforderlich ist, muss der Staat gewährleisten, dass dies auch ohne die Anstellung in einem Beamtenverhältnis auf Probe gewährleistet ist. \footcite[][Abs. 11]{noauthor_bverfge_nodate}

Auch wurde in den Sondervoten die Deutungshoheit des verfassungsfeindlichen Aspekts des Radikalenerlasses kritisiert. So sei einerseits nicht nur die Bezeichnung »verfassungsfeindlich« problematisch, weil es lediglich eine Definition des Begriffes »verfassungswidrig« gebe und so die Verfassungsfeindlichkeit für Interpretationen offen sei, sondern andererseits damit der Umstand, dass die Auslegung, welche Parteien verfassungswidrig sind, Aufgabe des Bundesverfassungsgerichts ist. \footcite[][S. 269]{edgar_wolfrum_verfassungsfeinde_2022}

Dabei dürfe jedoch auch aufgrund der Mitgliedschaft in einer vom Bundesverfassungsgericht verbotenen Partei eine Ablehnung der Bewerbung oder Entlassung erfolgen, nicht jedoch ausschließlich aufgrund der Zugehörigkeit zur Organisation, da diese nur ein Teil des Verhaltens sei. \footcite[][S. 269]{edgar_wolfrum_verfassungsfeinde_2022}

\subsubsection{Reaktionen auf das Urteil des Bundesverfassungsgerichts zum Radikalenerlass}

Während die Union sich durch das Urteil des Bundesverfassungsgerichts darin bestätigt sah, dass eine Mitgliedschaft in einer verfassungsfeindlichen Partei einen hinreichenden Grund für Zweifel an der Verfassungstreue darstellt, hielt die sozialliberale Koalition an ihrer Möglichkeit zur liberaleren Deutung des Beschlusses fest.  Da das Urteil insgesamt jedoch den Radikalenerlass duldete, fielen die Reaktionen Betroffener und Gegner erwartbar negativ aus. \footcite[][S. 269]{edgar_wolfrum_verfassungsfeinde_2022}

Die jeweils unterschiedlichen Vorstellungen der Parteien schafften in Kombination mit dem Urteil daraufhin keine Rechtssicherheit. Ebenso herrschte innerhalb der Landesregierungen, die sich ihrerseits ebenfalls auf den Radikalenerlass beriefen und stellenweise unterschiedliche Auslegungen vertraten, weiterhin - selbst innerhalb einzelner Regierungen - Uneinigkeit. So soll sich Landesinnenminister von Baden-Württemberg Karl Schiess in den wesentlichen Punkten bestätigt gesehen haben, weshalb keine Änderungen der Praxis vorzunehmen seien. Auch in einer internen Stellungnahme des Landesinnenministeriums verlautete man die Vereinbarkeit, gestand jedoch ein, dass das Urteil des Bundesverfassungsgerichts nicht in allen Punkten genaue Aussagen getroffen hatte. Brisanterweise erkannte man ebenfalls an, dass eine konträre Auslegung bei nicht allzugenauer Betrachtung plausibel sei. Daraufhin fügte man Empfehlungen an, wie dem in der Öffentlichkeit zu begegnen wäre.

Ebenfalls ergänzt wurde die Minimierung von Angriffsflächen, wonach die Einzelfallprüfung, besonders in Fällen, welche auf Basis einzelner Indizien für sich genommen noch nicht ausreichend für eine Ablehnung zu sehen sind, besondere Bedeutung erfahren soll. \footcite[][S. 271]{edgar_wolfrum_verfassungsfeinde_2022}

Aufgrund der uneindeutigen Interpretation von mehreren Seiten könne man daher das Urteil als »paradigmatisch für die damalige polarisierte Meinungs- und Konfliktlage« sehen, in welcher »selbst dem höchsten Richtergremium die Einigung auf unzweideutige und einheitliche Vorgaben für eine Gesetzgebung misslang.« \footcite[][S. 272]{edgar_wolfrum_verfassungsfeinde_2022}

\subsection{Zur Differenzierung der Positionen im öffentlichen Dienst und ihrer Relevanz im Hinblick auf die Gewähr der Verfassungstreue}

Die vom Bundesverfassungsgericht vertretene Position, dass die Gewähr der Verfassungstreue für alle im öffentlichen Dienst arbeitenden und Beamten erforderlich sei, wurde mehrfach angegriffen. So wurde durch das Urteil des Europäischen Gerichtshof für Menschenrechte einer vom Radikalenerlass erfassten Lehrerin zugesprochen, dass »das Ausüben der Tätigkeiten einer Gymnasiallehrerin (...) offensichtlich mit keinerlei Sicherheitsrisiken einher [gehe]«, sondern vielmehr lediglich die Gefahr der möglichen Indoktrinierung der Schüler bestehe. Darauf habe aber im Einzelfall die Lehrerin keineswegs abgezielt und ihre Arbeit sei positiv bewertet worden. Da sie also lediglich aufgrund ihrer aktiven Mitgliedschaft innerhalb der DKP entlassen worden war, ist dies als unverhältnismäßig und als Verstoß gegen die genannten Menschenrechte zu bewerten.

Auch die vom niedersächsischen Innenminister Richard Lehners vorgebrachten Versuche, die Regelanfrage der Bewerber beim Verfassungsschutz auf bestimmte Bereiche zu limitieren, fanden keinen fruchtbaren Boden. Mehrheitlich war man der Auffassung, dass das gegen den Gleichbehandlungsgrundsatz verstoße, da die »fdGO-Formeln der Beamtengesetze und Tarifverträge (...) für alle Bediensteten [gelten]«. \footcite[][S. 344]{rigoll_staatsschutz_2013} Legitimierweise lässt sich diese Notwendigkeit jedoch auf Basis der ausgeübten Tätigkeiten bezweifeln, da auch ein Postbote ein Beamter ist bzw. war, seine Tätigkeit und besonders deren Auswirkungen jedoch nicht von der Einhaltung der Verfassungstreue gleichermaßen bestimmt ist, wie die eines entscheidungstragenden Beamten in einer Behörde.

\subsection{Die Problematik des Rechtswegs}

Das Urteil des Europäischen Gerichtshofs für Menschenrechte zur Entlassung einer Lehrerin war in aller erster Linie jedoch nur ein Erfolg für die Klägerin. So waren etwa Schadensersatzzahlung für ähnlich Betroffene auf Basis des Urteils ausgeschlossen, weil das in der europäischen Menschenrechtskonvention beschriebene Verfahren vorsieht, dass »innerhalb eines Zeitraums von sechs Monaten nach dem Ergehen der letztinstanzlichen Entscheidung des jeweiligen Staates Individualbeschwerde bei der Kommission einzureichen ist, also zuvor auch der nationale Rechtsweg vollständig beschritten sein muss.« \footcite[][S. 276]{edgar_wolfrum_verfassungsfeinde_2022} Somit ist in jedem Einzelfall ein Prozess bis zum Bundesverfassungsgericht vonnöten, sowie eine anschließende erneute Einreichung auf europäischer Ebene. Doch selbst dann stellt sich die rechtliche Frage, inwiefern Entscheidungen des Europäischen Gerichtshofs für Menschenrechte überhaupt Vorrang vor den Gerichtsurteilen der Einzelstaaten haben. \footcite[][S. 277]{edgar_wolfrum_verfassungsfeinde_2022}

\section{Das Ende des Radikalenerlasses}
Die zunehmende Kritik aus In- und Ausland bewegte später die Regierung unter Helmut Schmidt die Regelungen abzuschwächen. So wurden 1979 zuerst neue Regelungen auf Bundesebene verabschiedet, wonach nur bei konkreten Verdachtsmomenten beim Verfassungschutz eine Abfrage erfolgen sollte - die Regelabfrage wurde zur Ausnahme. Sukzessive wandten sich die Bundesländer vom Radikalenerlass ab und auch Bayern löste sich 1991 als letztes Bundesland vom Beschluss. \footcite[][]{bildung_vor_nodate}

\newpage

\printbibliography

\newpage
\section*{Eidesstattliche Erklärung}

Hiermit versichere ich, dass ich die vorliegende Leistung selbstständig verfasst und keine anderen als die angegebenen Quellen und Hilfsmittel benutzt habe, alle Ausführungen, die anderen Schriften wörtlich oder sinngemäß entnommen wurden, kenntlich gemacht sind und die Arbeit in gleicher oder ähnlicher Fassung noch nicht Bestandteil einer Studien- oder Prüfungsleistung war.
\\[3em]
------------------------\\
Patrick Zauner\\
09. August 2022

\end{document}